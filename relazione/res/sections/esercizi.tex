\newpage
\section{Esercizi}

\subsection{Esercizio D - Trace equivalence}

\textbf{Dimostrare che la trace equivalence è una congruenza per il CCS.}

Dati 2 processi, P $\Re$ Q, $\Re$ è una congruenza sse: $\forall$ C[ ], $\quad C[P] \quad \Re \quad C[Q].$

Occorre dimostrare che: $P \sim_t Q \implies \forall$ C[ ], $\quad C[P] \sim_t C[Q]$.

\subsubsection{Dimostrazione per casi}

In ogni caso, l'ipotesi è che $Traces(P) = Traces(Q)$.

Di seguito userò la seguente convenzione: $\alpha.Traces(X) = \{ \alpha.x_1...x_k | \alpha.X \xrightarrow[]{\alpha} X \xrightarrow[]{x_1} ... \xrightarrow[]{x_k} X_k, k \geq 0\}$

\paragraph{Contesto vuoto}

C[ ] $= \emptyset \implies C[P] = C[Q] = \emptyset \implies C[P] \sim_t C[Q]$.

\paragraph{Prefisso $\alpha$}

C[ ] $= \alpha ...$

$Traces(C[P]) = Traces(\alpha.P) = \alpha.Traces(P)$.

Per ipotesi Traces(P) = Traces(Q), quindi $\alpha.Traces(P) = \alpha.Traces(Q) = Traces(\alpha.Q)$.

L'equivalenza è preservata: $C[P] \sim_t C[Q]$.

\paragraph{Contesto non deterministico}

C[ ] $= ( \quad + R)$

Se si dimostra che $\forall$ P,R processi CCS $Traces(P + R) = Traces(P) \cup Traces(R)$ allora anche $Traces(Q + R) = Traces(Q) \cup Traces(R)$. Per ipotesi Traces(P) = Traces(Q), quindi $Traces(P + R) = Traces(Q + R)$.

\paragraph{($\subseteq$)}

Sia $t \in Traces(P + R)$. Ci sono 2 possibilità:

1 - P+R ha effettuato una transizione usando la regola SUM1:

\begin{prooftree}
\AxiomC{$P \xrightarrow[]{a_1} P'$}
\UnaryInfC{$P + R \xrightarrow[]{a_1} P'$}
\end{prooftree}

In questo caso $t \in a_1.Traces(P') \subseteq Traces(P)$ quindi $t \in Traces(P)$.

2 - P+R ha effettuato una transizione usando la regola SUM2:

\begin{prooftree}
\AxiomC{$R \xrightarrow[]{r_1} R'$}
\UnaryInfC{$P + R \xrightarrow[]{r_1} R'$}
\end{prooftree}

In questo caso $t \in r_1.Traces(R') \subseteq Traces(R)$ quindi $t \in Traces(R)$.

\paragraph{($\supseteq$)}

Sia $t \in Traces(P) \cup Traces(R)$. 

Allora t è una traccia di P o t è una traccia di R o di entrambi. 

Se $t \in Traces(P)$, P+R può effettuare una transizione usando la regola SUM1, così $t \in Traces(P+R)$.

Se $t \in Traces(R)$, P+R può effettuare una transizione usando la regola SUM2, così $t \in Traces(P+R)$.

Se la traccia t appartiene sia a P che R allora la regola che P + R segue per effettuare una transizione è indifferente.

\paragraph{Contesto parallelo}

C[ ] = | R.

Occorre dimostrare che Traces(P|R) = Traces(Q|R) \\

Dapprima dimostro il seguente fatto: dati tre processi CCS A,B e C

\begin{equation}
\begin{aligned}
\label{eq:th}
Traces(A) \subseteq Traces(B) \implies (\forall n \in N, |t| = n, \\
t \in Traces(A|C) \implies t \in Traces(B|C)).
\end{aligned}
\end{equation} \\

Poiché Traces(P) = Traces(Q), l'inclusione vale in entrambi i versi e

si conclude che $\forall n \in N, |t| = n, t \in Traces(P|R) \iff t \in Traces(Q|R)$,

che equivale a dire che $Traces(P|R) = Traces(Q|R).$ \\

Prima di procedere con la dimostrazione di \ref{eq:th}, si dimostra il seguente
lemma:

Siano H, K due processi CCS:

\begin{equation}
\begin{aligned}
\label{eq:lem}
Traces(H) = Traces(K) \implies (H \xrightarrow[]{\alpha} H' \implies \\
Traces(H') \subseteq Traces(\sum \{ K' | K \xrightarrow[]{\alpha} K' \}) \forall \alpha)
\end{aligned}
\end{equation} \\

Dimostrazione \ref{eq:lem} \\

$ Traces(\sum \{ K' | K \xrightarrow[]{\alpha} K' \}) = \bigcup_{K \xrightarrow[]{\alpha} K'} Traces(K') = 
\{ t' | t = \alpha.t' , t \in Traces(K) \} = \{ t' | t = \alpha.t' , t \in Traces(H) \}. $

$ Traces(H') \subseteq \{ t' | t = \alpha.t' , t \in Traces(H) \} = Traces(\sum \{ K' | K \xrightarrow[]{\alpha} K' \}) $ \\

Questo lemma verrà usato nella dimostrazione di \ref{eq:th}. \\

Dimostrazione \ref{eq:th} (per induzione sulla lunghezza della traccia t) \\

\textbf{Caso base}: |t| = 0 ($\epsilon \in Traces(P|R) \implies \epsilon \in Traces(Q|R)$) : vero \\

\textbf{Ipotesi induttiva}: l'enunciato vale su una qualsiasi traccia t, |t| = n. \\

\textbf{Caso induttivo} |t| = n + 1.

Sia $t \in Traces(P|R)$. Ci sono tre possibilità: \\

1 - P|R ha effettuato una transizione usando la regola COM1:

\begin{prooftree}
\AxiomC{$P \xrightarrow[]{a_1} P'$}
\UnaryInfC{$P | R \xrightarrow[]{a_1} P' | R$}
\end{prooftree}

$t = a_1.t', |t'| = n, t' \in Traces(P'|R)$ \\

Ipotesi induttiva applicata a $t'$: \\

Per il lemma \ref{eq:lem}, $Traces(P') \subseteq  Traces(\sum \{ Q' | Q \xrightarrow[]{a_1} Q' \}) \implies \\
t' \in Traces(P'|R) \implies t' \in Traces(\sum_{Q \xrightarrow[]{a_1} Q'} Q'|R).$ \\

Da questo segue che $t \in a_1.Traces(\sum_{Q \xrightarrow[]{a_1} Q'} Q'|R) = a_1.Traces(\sum_{Q \xrightarrow[]{a_1} Q'} (Q'|R)) \\
= Traces(a_1.(\sum_{Q \xrightarrow[]{a_1} Q'} (Q'|R))) = Traces(\sum_{Q \xrightarrow[]{a_1} Q'} a_1.(Q'|R)) = \\
\bigcup_{Q \xrightarrow[]{a_1} Q'} Traces(a_1.(Q'|R))$\\

A questo punto, poiché il processo Q|R è in grado di muoversi con un passo $a_1$ in un processo $Q'|R$ applicando la regola COM1,
si conclude che $t \in Traces(Q|R).$ \\

2 - P|R ha effettuato una transizione usando la regola COM2:

\begin{prooftree}
\AxiomC{$R \xrightarrow[]{r_1} R'$}
\UnaryInfC{$P | R \xrightarrow[]{r_1} P| R'$}
\end{prooftree}

$t = r_1.t', |t'| = n, t' \in Traces(P|R')$ \\

Ipotesi induttiva applicata a $t'$: \\

Per il lemma \ref{eq:lem}, $Traces(P) \subseteq  Traces(Q) \implies t' \in Traces(P|R') \implies t' \in Traces(Q|R').$ \\

Da questo segue che $t \in r_1.Traces(Q|R') \implies t \in Traces(r_1.(Q|R')).$

A questo punto, poiché il processo Q|R è in grado di muoversi con un passo $r_1$ applicando la regola COM2,
si conclude che $t \in Traces(Q|R).$ \\

3 - P|R ha effettuato una transizione usando la regola COM3:

\begin{prooftree}
\AxiomC{$P \xrightarrow[]{a_x} P'$}
\AxiomC{$R \xrightarrow[]{\overline{a_x}} R'$}
\BinaryInfC{$P | R \xrightarrow[]{\tau} P'| R'$}
\end{prooftree}

$t = \tau.t', |t'| = n, t' \in Traces(P'|R')$ \\

Ipotesi induttiva applicata a $t'$: \\

Per il lemma \ref{eq:lem}, $Traces(P') \subseteq  Traces(\sum \{ Q' | Q \xrightarrow[]{a_x} Q' \}) \implies \\
t' \in Traces(P'|R') \implies t' \in Traces(\sum_{Q \xrightarrow[]{a_x} Q'} Q'|R').$ \\

Da questo segue che $t \in \tau.Traces(\sum_{Q \xrightarrow[]{a_x} Q'} Q'|R') = \tau.Traces(\sum_{Q \xrightarrow[]{a_x} Q'} (Q'|R')) \\
= Traces(\tau.(\sum_{Q \xrightarrow[]{a_x} Q'} (Q'|R'))) = Traces(\sum_{Q \xrightarrow[]{a_x} Q'} \tau.(Q'|R')) = \\
\bigcup_{Q \xrightarrow[]{a_x} Q'} Traces(\tau.(Q'|R'))$ \\

A questo punto, poiché il processo Q è in grado di sincronizzarsi con il processo R,
Q|R può muoversi con un'azione $\tau$ in un processo $Q'|R'$, quindi si può concludere che $t \in Traces(Q|R).$

\paragraph{Contesto con ri-etichettatura}

C[ ] =  [f]

\begin{equation*}
f: Act \to Act \quad t.c \quad f(\tau) = \tau \land \forall a \in \mathcal{L}, f(\overline{a}) = \overline{f(a)}
\end{equation*}

Dimostrare che Traces(P[f]) = Traces(Q[f]):

\paragraph{($\subseteq$)}

Sia $tp \in Traces(P), tp = a_1.a_2.a_3...$. Allora $tp' = f(a_1).f(a_2).f(a_3)... \in Traces(P[f]).$
Per ipotesi, $tp \in Traces(Q)$. Allora $tp' = f(a_1).f(a_2).f(a_3)... \in Traces(Q[f]).$

\paragraph{($\supseteq$)}: analogo.

\paragraph{Contesto con restrizione}

C[ ] = $\setminus L, L \subseteq \mathcal{L}$

$Traces(C[P]) = Traces(P \setminus L) = Traces(P) \setminus \{ t = ...a_k... | a_k \in L\}$

$Traces(C[Q]) = Traces(Q \setminus L) = Traces(Q) \setminus \{ t = ...a_k... | a_k \in L\}$

$Traces(P) = Traces(Q) \implies Traces(P) \setminus \{ t = ...a_k... | a_k \in L\} = Traces(Q) \setminus \{ t = ...a_k... | a_k \in L\}$

\subsection{Esercizio U}

\textbf{Mostrare la coincidenza di string bisimilarity e strong bisimilarity}

$\Re$ è \textbf{string bisimulation} se:\\

se P $\Re$ Q e P $\xrightarrow[]{a_1} ... \xrightarrow[]{a_n}$ P', allora Q $\xrightarrow[]{a_1} ... \xrightarrow[]{a_n}$ Q' e P' $\Re$ Q';

se P $\Re$ Q e Q $\xrightarrow[]{a_1} ... \xrightarrow[]{a_n}$ Q', allora P $\xrightarrow[]{a_1} ... \xrightarrow[]{a_n}$ P' e P' $\Re$ Q'. \\

P $\sim_{string}$ Q se $\exists \Re$ string bisimulation tale che P $\Re$ Q.

Dimostrare che $P \sim Q \Leftrightarrow P \sim_{string} Q $.

Dimostro che i due tipi di bisimulazione sono equivalenti, così devono esserlo anche le bisimilarità.

\paragraph{($\Longleftarrow$)}

Se P $\Re_{string}$ Q, allora applicando la definizione per una qualsiasi sequenza di azioni $\sigma$ di lunghezza 1, $\sigma = \alpha \in \mathcal{L}:$ \\

se P $\xrightarrow[]{\alpha} P'$, allora Q $\xrightarrow[]{\alpha} Q'$ e $P'$ $\Re_{string}$ $Q'$ + duale \\

che coincide con la definizione di strong bisimulation.

\paragraph{($\Longrightarrow$)}

Idea: per n = 1 è evidente, per n > 1 occorre applicare la definizione di $\sim$ in più passi.

Dimostrazione per induzione sulla lunghezza della sequenza di azioni $\sigma$.

\paragraph{$|\sigma| = 1 \quad \sigma = \alpha \in \mathcal{L}$}

In questo caso la definizione di string bisimulation e strong bisimulation coincidono.

\paragraph{$|\sigma| > 1$}

Ipotesi induttiva: $\forall \Re$ strong bisimulation, P $\Re$ Q $\implies \exists \Re_{string}$ per una sequenza di azioni $\sigma,
|\sigma| = n$ tale che P $\Re_{string}$ Q.

Sia $|\sigma| = n + 1$ e $\Re$ strong bisimulation.

P $\Re$ Q $\implies$ (per ipotesi induttiva) $\exists \Re_{string}$ per una sequenza di azioni $\sigma,
|\sigma| = n$ tale che P $\Re_{string}$ Q.

Inoltre P $\Re$ Q $\implies$ $P_n$ $\Re$ $Q_n$. In questo caso vale: \\

se $P_n \xrightarrow[]{\alpha} P_{n+1}$, allora $Q_n \xrightarrow[]{\alpha} Q_{n+1}$ e $P_{n+1}$ $\Re$ $Q_{n+1}$ + duale \\

Così la lunghezza della sequenza di azioni $\sigma$ relativa alla string bisimulation $\Re_{string}$ può essere estesa a n+1.
