\newpage
\section{Approfondimento - Concorrenza in Rust}

\subsection{Overview}

\begin{itemize}
  \item Linguaggio nato nel 2010, sviluppato da Mozilla
  \item Rivolto alla programmazione di sistem
  \item Linguaggio compilato
  \item Linguaggio staticamente tipato (utilizza type inference)
  \item Caratteristiche dai paradigmi di programmazione procedurale e funzionale
  \item Sintassi molto simile a C/C++

\end{itemize}

\subsection{Memory Safety}

\textit{Nota: per brevità, tralascio la sintassi di Rust e i suoi costrutti
principali; mi dedicherò solamente alla parte di concorrenza} \\

Rust è un linguaggio che tiene particolarmente alla cosiddetta \textbf{memory
safety}, una proprietà che assicura la protezione da bugs e vulnerabilità quando
si ha a che fare con la memoria (errori di accesso, variabili non inizializzate,
memory leaks, \dots).

Per far fronte a queste problematiche, utilizza un sistema chiamato
\textbf{ownership system}, che ha lo scopo di segnalare al programmatore un
utilizzo non sicuro della memoria a tempo di compilazione.

Accanto alla memory safety, un altro degli obiettivi di Rust è quello di essere
un linguaggio in grado di gestire codice concorrente in modo sicuro.

L'ambito della concorrenza è storicamente complesso ed è più facile commettere
degli errori. Rust affronta la questione facendo affidamento sul suo ownership
system. Dopotutto alcuni errori sono comuni (le data races sono errori logici di
accesso alla memoria ed errori di concorrenza).

\subsection{Regole del sistema di Ownership}

\begin{itemize}
  \item Ogni valore è legato a un owner, che può essere una variabile, una funzione
o un thread.
  \item Ogni valore può avere un solo owner alla volta.
  \item Quando l'owner di un valore va out of scope, il valore viene scartato.
\end{itemize}

\subsection{Regole di borrowing}

Ogni borrow (prestito dell'ownership) deve avere uno scope \textbf{non più grande} di
quello dell’owner (non può vivere più a lungo).

Inoltre, possono valere le seguenti condizioni \textbf{in mutua esclusione}:

\begin{itemize}
\item \textbf{Uno o più riferimenti immutabili} a una risorsa
\item Esattamente \textbf{un riferimento mutabile} a una risorsa
\end{itemize}

\subsection{Creazione di un thread}

Un nuovo thread può essere creato tramite la funzione \texttt{thread::spawn}:

\begin{minted}{rust}
use std::thread;

thread::spawn(move || {
    // some work here
});
\end{minted}

Un thread genitore può attendere il completamento dell'esecuzione del thread
figlio: una chiamata a \texttt{thread::spawn} produce un \texttt{JoinHandle},
che fornisce una funzione join per l'attesa:

\begin{minted}{rust}
use std::thread;

let child = thread::spawn(move || {
    // some work here
});
// some work here
let res = child.join();
\end{minted}

La funzione join ritorna un \texttt{Result} contenente un valore di tipo
\texttt{Ok} se non ci sono stati errori, oppure \texttt{Err}.

\subsection{Prevenzione delle data races}

\textbf{Data races}: accessi multipli \textbf{non sincronizzati} alla
medesima posizione in memoria di cui almeno uno di essi è in scrittura.

Il sistema di ownership di Rust, combinato alle regole di borrowing,
sembra offrire una sincronizzazione implicita agli accessi in memoria,
che previene qualsiasi data race.

Come esempio il seguente programma Rust, contenente una data race
in altri linguaggi, non compila:

\begin{minted}{rust}
use std::thread;
use std::time:: Duration;

fn main() {
    let mut data = vec![1, 2, 3]; // shared mutable state!

    for i in 0..3 {
        thread::spawn(move || {
            data[0] += i; // operation on data
        });
    }

    thread::sleep(Duration::from_millis(50));
}
\end{minted}

L’errore è il seguente:

\begin{minted}{rust}
8:17 error: capture of moved value: data
        data[0] += i;
        ^~~~
\end{minted}

Rust sa che questo comportamento non è sicuro. Se avessimo un riferimento a data
in ogni thread, e i threads prendessero l’ownership del riferimento, avremmo
tre owners. Questo errore è chiamato \textbf{use of moved value}, causato dal
tentativo di utilizzare un dato di cui non si detiene l'ownership.

\paragraph{Message Passing:}

Un approccio alla concorrenza che sembra crescere in popolarità è il
message passing, dove i threads (o attori) comunicano inviando fra loro
messaggi che contengono dati.
A tal proposito, lo slogan di Go dice: \\

``Do not communicate by sharing memory; instead, share memory by
communicating.'' - Effective Go \\

I canali sono lo strumento principale per raggiungere questo obiettivo.

In Rust è presente il modulo \texttt{std::sync::mpsc} (Multi-producer,
single-consumer) che fornisce delle primitive per la comunicazione
message-based, che può avvenire per mezzo dei seguenti tipi:

\begin{itemize}
  \item Sender
  \item SyncSender
  \item Receiver
\end{itemize}

Un \texttt{Sender} o un \texttt{SyncSender} può essere usato per
inviare dati a un \texttt{Receiver}.

Questi canali possono essere:

\begin{itemize}
  \item asincroni, unbounded (si può pensare che abbiano un buffer infinito).
La tupla ritornata dalla primitiva \texttt{std::sync::mpsc::channel} avrà
tipo (Sender, Receiver), e tutti gli invii sono non bloccanti.

  \item sincroni, bounded, bloccanti. La tupla ritornata dalla primitiva

\texttt{std::sync::mpsc::sync\_channel} avrà tipo (SyncSender, Receiver),
in cui il buffer per i messaggi pendenti sarà di una dimensione fissata.
Tutti gli invii sono bloccanti (se il buffer è pieno, attende fino a che non si libera).
\end{itemize}

Questa scelta è diversa da Go, dove le send e le receive sono,
di base, bloccanti (a meno di una clausola default all'interno di una
select, o la specifica della dimensione del buffer).

In un'implementazione del problema dei prigionieri o dei 5 filosofi,
l'utilizzo di canali con un \texttt{SyncSender} è d'obbligo, altrimenti,
ad esempio, il counterPrisoner potrebbe accendere la lampadina
infinite volte e poi dichiarare freedom anche se non tutti i
prigionieri sono stati nella stanza.

Esempio d'uso dei canali in Rust:

\begin{minted}{rust}

use std::thread;
use std::sync::mpsc;

fn main() {
    let (tx, rx) = mpsc::channel();

    thread::spawn(move || {
        let val = String::from("hi");
        tx.send(val).unwrap();
    });

    let received = rx.recv().unwrap();
    println!("Got: {}", received);
}
\end{minted}

\textbf{Come i canali interagiscono con il meccanismo di ownership}

Nel seguente esempio si tenta di usare un valore nel thread dopo averlo inviato
sul canale:

\begin{minted}{rust}

use std::thread;
use std::sync::mpsc;

fn main() {
    let (tx, rx) = mpsc::channel();

    thread::spawn(move || {
        let val = String::from("hi");
        tx.send(val).unwrap();
        println!("val is {}", val); // moved value!
    });

    let received = rx.recv().unwrap();
    println!("Got: {}", received);
}
\end{minted}

Questa è una cattiva idea: una volta che abbiamo inviato il valore a un altro
thread, quel thread può modificarlo o prima che il sender
lo utilizzi di nuovo. Questo potrebbe causare errori dovuti alla non-esistenza
o inconsistenza dei dati.

Se proviamo a compilare l'esempio, Rust segnala:


\begin{minted}{rust}

error[E0382]: use of moved value: val
  --> src/main.rs:10:31
   |
9  |         tx.send(val).unwrap();
   |                 --- value moved here
10 |         println!("val is {}", val);
   |                               ^^^ value used here after move
   |
   = note: move occurs because val has type std::string::String,
which does not implement the Copy trait

\end{minted}

Questo errore di concorrenza ha causato un errore di compilazione.
Send prende l'ownership dei suoi parametri e sposta (moves) il valore
in modo da passare l'ownerhip al ricevente.

Questo significa che non è possibile utilizzare il valore di nuovo dopo averlo
inviato.

In questo contesto, message passing è molto simile a single ownership: se solo
un thread alla volta può usare della memoria, non c'è il rischio di data race
(in alternativa, si può sempre passare una copia del valore).

Si nota che il sistema di Ownership di Rust e le regole di borrowing prevengono
solamente le data races e non tutte le race conditions che possono verificarsi
durante l'esecuzione di un programma concorrente (ad esempio, i deadlocks possono
comunque verificarsi).

\subsection{Aspetti positivi}

\begin{itemize}
  \item Commenti dettagliati da parte del compilatore in risposta agli errori
che aiutano a individuare il problema
  \item Niente npe, niente dangling pointers
  \item Niente data races
\end{itemize}

\subsection{Aspetti negativi}

\begin{itemize}
  \item Lunghi tempi di apprendimento
  \item Lotta contro il compilatore (difficile comprensione del sistema di
ownership e delle regole di borrowing) da parte dei programmatori
\end{itemize}

\subsection{Aspetti neutri}

\begin{itemize}
  \item Linguaggio giovane (nato nel 2010)
  \item Librerie, guide e tutorial in fase di espansione
  \item Niente eccezioni
  \item Da valutare la scelta in base a due indicatori: performances e memory safety.
\end{itemize}
