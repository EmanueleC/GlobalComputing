\newpage
\section{Prisoners game}

\paragraph{Testo del problema:}
50 prigionieri si trovano rinchiusi in celle separate e di tanto in tanto
vengono portati singolarmente in una stanza speciale (senza un ordine,
possibilmente più volte di seguito).

La stanza è completamente vuota a eccezione di una lampadina che può accendersi
o spegnersi (la luce non è visibile dall'esterno e non si può rompere).

In ogni momento, ciascun prigioniero può dire che tutti sono stati nella stanza
con la lampadina. Se è vero, allora tutti verranno liberati (ma se non lo è
resteranno imprigionati per sempre).

Prima che la sfida cominci, i prigionieri possono stabilire insieme un
\textbf{protocollo} da seguire per vincere il gioco.

Attenzione: \textit{lo stato iniziale della lampadina non è noto}.

\subsection{Una possibile soluzione}

Sia n il numero di prigionieri.

Il protocollo che i prigionieri devono rispettare è il seguente:

\paragraph{Protocollo - Soluzione:}

Viene scelto un \textbf{CounterPrisoner}, che avrà una speciale responsabilità.

Tutti gli altri prigionieri si devono comportare nel modo seguente: ognuno di
loro spegne la luce solo le prime due volte che la trova accesa e la lascia
così com'è tutte le altre.

Il \textbf{CounterPrisoner} accende la luce se la trova spenta e tiene il conto
del numero di volte che lo fa. Quando raggiunge \textbf{2*(n - 1)} può dire che tutti
i prigionieri sono stati nella stanza.

In una versione alternativa del problema, quella in cui si conosce lo stato
iniziale della lampadina, CounterPrisoner dovrebbe contare solo la metà
delle volte \textbf{n - 1}.

\subsection{Modellazione del protocollo in CCS}

Le entità principali individuate sono:

\begin{itemize}
\item $Prisoner_i, \quad i = [1, n-1]$
\item \textbf{Counter prisoner}
\item \textbf{Bulb/Light}
\item Sistema: \textbf{Sys}
\end{itemize}

e si traducono tutte in processi CCS, che comunicano fra loro attraverso i
canali \textbf{on e off} (i prigionieri devono sincronizzarsi con la lampadina).

Sono stati aggiunti anche dei canali enter e freedom, utili per esprimere delle
proprietà del sistema (analogamente a quanto è stato fatto nella traduzione in
CCS dell'algoritmo di Peterson, con le etichette $enter_i$/$exit_i$).

Enter e freedom sono canali pubblici.

In questa prima modellazione i prigionieri si ``fermano'' dopo aver svolto il ruolo
imposto dal protocollo (dopo aver spento due volte la lampadina).

Una conseguenza di questa decisione è l'assenza di livelock: non è possibile che un
prigioniero entri per sempre nella stanza e che gli altri rimangano ad aspettare.

Di seguito viene dato un esempio del sistema con 4 prigionieri.
Il contatore dovrà accendere 2*(4 - 1) = 6 volte la lampadina per essere certo che tutti
i prigionieri siano entrati nella cella. \\

$Prisoner1 = enter.\overline{off}.Prisoner1Off;$

$Prisoner1Off = \overline{off}.0;$ \\

$Prisoner2 = enter.\overline{off}.Prisoner2Off;$

$Prisoner2Off = \overline{off}.0;$ \\

$Prisoner3 = enter.\overline{off}.Prisoner3Off;$

$Prisoner3Off = \overline{off}.0;$ \\


$CounterPrisoner = \overline{on}.CounterPrisoner1;$

$CounterPrisoner1 = \overline{on}.CounterPrisoner2;$

$CounterPrisoner2 = \overline{on}.CounterPrisoner3;$

$CounterPrisoner3 = \overline{on}.CounterPrisoner4;$

$CounterPrisoner4 = \overline{on}.CounterPrisoner5;$

$CounterPrisoner5 = \overline{on}.\overline{freedom}.0;$ \\

$Light = LightOn + LightOff;$

$LightOn = off.LightOff;$

$LightOff = on.LightOn;$ \\

$Sys = (Prisoner1 | Prisoner2 | Prisoner3 | CounterPrisoner | Light) \setminus \{on, off\};$ \\

La seguente specifica Spec serve a indicare il comportamento atteso della
soluzione:

\begin{equation*}
Spec = enter.enter.enter.\overline{freedom}.0;
\end{equation*}

Solamente dopo che i tre prigionieri sono entrati nella stanza, è possibile che
il CounterPrisoner dichiari $\overline{freedom}$.

La specifica si preoccupa di descrivere il comportamento del sistema ad alto
livello, quindi la relazione da usare non può essere la bisimilarità forte, ma
quella debole:

\begin{equation*}
Sys \approx Spec
\end{equation*}

ed è soddisfatta.

Non è un caso che le etichette $enter_i$ siano state inserite prima che i
prigionieri eseguano $\overline{off}$: se fossero state inserite dopo, la 
relazione non sarebbe più stata soddisfatta.

Infatti c'è un caso in cui il counterPrisoner può dire $\overline{freedom}$
prima che tutti i prigionieri abbiano spento due volte la luce, ed è quello
in cui lo stato iniziale della lampadina sia spento (ciò non causa problemi
perché tutti i prigionieri devono comunque spegnerla almeno una volta). \\

Vale la seguente proprietà (\textit{Even(<$\overline{freedom}$>tt)}):

\begin{equation*}
Sys \models X \quad
  X \quad min= <\overline{freedom}>tt \lor (<->tt \land [-]X)
\end{equation*}

Il CounterPrisoner prima o poi dirà che tutti i prigionieri sono stati nella
stanza e dichiarerà freedom. \\

La proprietà di \textit{livelock = Pos(Cycle)} (esiste uno stato raggiungibile
in cui c'è un loop interno):

\begin{equation*}
Sys \models Livelock
\end{equation*}
\begin{equation*}
Livelock \quad min= Cycle \lor <->Livelock; \quad
Cycle \quad max= <tau>Cycle;
\end{equation*}

non è soddisfatta. \\

Vale la proprietà di fine esecuzione: dopo che il CounterPrisoner ha dichiarato
freedom, nessun prigioniero può fare una transizione enter.

Dapprima si verifica che la proprietà $Pos(<\overline{freedom}><enter>tt)$:

\begin{equation*}
Sys \models Y
\end{equation*}
\begin{equation*}
Y \quad min= <\overline{freedom}><enter>tt \lor <->Y;
\end{equation*}

non sia soddisfatta.

Una volta effettuata la traduzione in $\neg Inv(\neg <\overline{freedom}><enter>tt>)$,
che corrisponde a $\neg Inv([\overline{freedom}][enter]ff)$, si verifica la sua negazione:

\begin{equation*}
Sys \models Z
\end{equation*}
\begin{equation*}
Z \quad max= [\overline{freedom}][enter]ff \land [-]Z;
\end{equation*}

ed è soddisfatta. \\

\textbf{Nota}: se c'è un solo prigioniero, allora questo dovrà essere per forza il
counter prisoner (che appena entra nella stanza dichiara freedom).
In questo modello, conta fino a 2*(n - 1) che è 0 se c'è un solo prigioniero.

\subsection{Possibilità di livelock}

Una seconda modellazione, che tenga conto della possibilità per un prigioniero di entrare
più volte nella stanza con la lampadina, inserisce un ciclo interno in ogni processo
prigioniero, dopo due successivi spegnimenti della lampadina.

In questo modo un livelock è possibile, anche se questo non preclude l'esistenza di un
cammino che porti il counterPrisoner a dichiarare freedom. \\

$Prisoner1 = enter.\overline{off}.Prisoner1Off;$

$Prisoner1Off = \overline{off}.Wait;$ \\

$Prisoner2 = enter.\overline{off}.Prisoner2Off;$

$Prisoner2Off = \overline{off}.Wait;$ \\

$Prisoner3 = enter.\overline{off}.Prisoner3Off;$

$Prisoner3Off = \overline{off}.Wait;$ \\

$Wait = tau.Wait;$ \\

Il sistema è ancora debolmente bisimile alla specifica precedente. Le proprietà che cambiano
sono quelle di livelock (che ovviamente diventa vera) ed eventually freedom, che diventa
falsa.

Invece, vale la proprietà più debole $Pos(<\overline{freedom}>tt)$: esiste uno stato raggiungibile
in cui il counterPrisoner dichiara freedom.

\subsection{Implementazione in Go}

L'implementazione in Go ricalca la definizione dei processi CCS, a eccezione
della lampadina.

La lampadina possiede uno stato iniziale non noto; questo si riflette
nell'impossibilità di renderla ``stateless'' e nell'inserimento della scelta
if-then-else basata sul suo stato corrente.

\begin{minted}{go}
// Bulb behaviour
func (self *Bulb) Run () {
  for {
    if self.on {
      <- self.Off
      fmt.Printf("Light switched off \n")
      self.on = false;
    } else {
      <- self.On
      fmt.Printf("Light switched on \n")
      self.on = true;
    }
  }
}
\end{minted}

per quanto riguarda i canali On e Off, sono canali bidirezionali di interi
(l'invio di un intero su On/Off corrisponde all'accensione/spegnimento).

Ogni processo è stato trasformato in una go routine, lanciata nel main.


